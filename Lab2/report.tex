\documentclass{ctexart}
\usepackage{graphicx}
\usepackage{caption}
\usepackage{float}
\usepackage{amsmath}
\usepackage{fancyhdr}
\usepackage{xunicode-addon}
\usepackage{booktabs}
\usepackage{listings}
\usepackage{hyperref}
\usepackage{longtable}
\usepackage[a4paper,hmargin=1.25in,vmargin=1in]{geometry}
% !TeX program = xelatex
\lstdefinestyle{mystyle}{
  basicstyle=\ttfamily\footnotesize,
  breakatwhitespace=false,         
  breaklines=true,                 
  captionpos=b,                    
  keepspaces=true,                 
  numbers=left,                    
  numbersep=5pt,                  
  showspaces=false,                
  showstringspaces=false,
  showtabs=false,                  
  tabsize=2
}

\lstset{style=mystyle}

\title{\begin{figure}[H]
	\centering 
	\includegraphics[height=7cm,width=14cm]{E:/Pictures/中科大.jpg}
	\end{figure}\Huge\textbf{图书馆管理系统课程设计报告}}
\date{}
\punctstyle{banjiao} 
\pagestyle{fancy}
	\fancyhead[C]{\LARGE\textbf{图书馆管理系统}}
	\fancyhead[L]{}
	\fancyhead[R]{}
	\fancyfoot[C]{\thepage}
\begin{document}
	\maketitle
	\thispagestyle{empty}
	
	\[\makebox{\Large{姓名:\underline{\makebox[5cm]{高茂航、李宇湘}}}}\]
	
	$$\makebox{\Large{日期:\underline{\makebox[5cm]{2024.7.20}}}}$$
	
	\clearpage

	\pagenumbering{arabic}

	\section{}
	
	
	
	\section{}

	
	
	\section{数据库概念结构设计}
	\begin{figure}[H]
		\centering 
		\includegraphics[height=10cm,width=12cm]{ER.png}
		\caption{ER图}
	\end{figure}

	\section{数据库逻辑结构设计(models.py)}
	\subsection{数据库表结构}
	\begin{longtable}{p{3.5cm}p{3.5cm}p{5.5cm}}
		\toprule
		\textbf{表名 (表名含义)} & \textbf{列名 (注释)} & \textbf{数据类型} \\
		\midrule
		\endhead
		dzTable (读者表) & \underline{dzid} (读者ID) & AutoField \\
		& psw (密码) & CharField(max\_length=256) \\
		& xm (姓名) & CharField(max\_length=10) \\
		\midrule
		tsglyTable (图书管理员表) & \underline{glyid} (管理员ID) & CharField(max\_length=10) \\
		& psw (密码) & CharField(max\_length=256) \\
		& xm (姓名) & CharField(max\_length=10) \\
		\midrule
		smTable (书目表) & \underline{isbn} & CharField(max\_length=50) \\
		& sm (书名) & CharField(max\_length=50) \\
		& zz (作者) & CharField(max\_length=50) \\
		& cbs (出版社) & CharField(max\_length=50) \\
		& cbny (出版时间) & DateTimeField \\
		& count (数量) & IntegerField(default=0) \\
		\midrule
		tsTable (图书实例表) & \underline{tsid} (图书ID) & AutoField \\
		& isbn (对应书目) & ForeignKey(smTable) \\
		& cfwz (存放位置) & CharField(max\_length=20) \\
		& zt (状态) & CharField(max\_length=20) \\
		& jbr (入库该书管理员工号) & ForeignKey(tsglyTable) \\
		\midrule
		BookReview (书评表) & \underline{dzid} (读者ID) & ForeignKey(dzTable) \\
		& \underline{isbn}& ForeignKey(smTable) \\
		& score (评分) & IntegerField(validators=[0,10]) \\
		& comment (评论) & TextField(max\_length=300, null=True) \\
		& comment\_time (评论时间) & DateTimeField(auto\_now\_add=True) \\
		\midrule
		jsTable (借书表) & \underline{dzid} (读者ID) & ForeignKey(dzTable) \\
		& \underline{tsid} (图书ID) & ForeignKey(tsTable, null=True) \\
		& \underline{jysj} (借阅时间) & DateTimeField \\
		& yhsj (应还时间) & DateTimeField \\
		& ghsj (归还时间) & DateTimeField(null=True) \\
		& is\_valid (记录是否有效) & BooleanField(default=True) \\
		\bottomrule
		\end{longtable}
		\subsection{具体分析}

		\subsubsection{dzTable (读者表)}
\begin{itemize}
    \item \textbf{结构}: 包含读者ID(主码)、密码、姓名。
    \item \textbf{解释}: 此表用于存储图书馆读者的基本信息。
    \item \textbf{3NF分析}: 满足3NF,因为每个非主属性完全函数依赖于主码(读者ID),不存在传递依赖或部分依赖。
    \item \textbf{BCNF分析}: 满足BCNF,由于不存在非主属性对主码的部分或传递依赖,同时每个候选码也是超码。
    \item \textbf{4NF分析}: 满足4NF,因为表中不存在多值依赖。
\end{itemize}

\subsubsection{tsglyTable (图书管理员表)}
\begin{itemize}
    \item \textbf{结构}: 包含管理员ID(主码)、密码、姓名。
    \item \textbf{解释}: 此表用于存储图书管理员的基本信息。
    \item \textbf{3NF分析}: 满足3NF,每个非主属性完全函数依赖于主码(管理员ID),不存在传递依赖或部分依赖。
    \item \textbf{BCNF分析}: 满足BCNF,因为不存在非主属性对主码的部分或传递依赖,且每个候选码也是超码。
    \item \textbf{4NF分析}: 满足4NF,因为表中不存在多值依赖。
\end{itemize}

\subsubsection{smTable (书目表)}
\begin{itemize}
    \item \textbf{结构}: 包含ISBN(主码)、书名、作者、出版社、出版时间、数量。
    \item \textbf{解释}: 此表记录了馆藏每个ISBN的书目的详细信息。
    \item \textbf{3NF分析}: 满足3NF,因为每个非主属性完全函数依赖于主码(ISBN),不存在传递依赖或部分依赖。
    \item \textbf{BCNF分析}: 满足BCNF,因为不存在非主属性对主码的部分或传递依赖,且每个候选码也是超码。
    \item \textbf{4NF分析}: 满足4NF,因为表中不存在多值依赖。
\end{itemize}

\subsubsection{tsTable (图书实例表)}
\begin{itemize}
    \item \textbf{结构}: 包含图书ID(主码)、该书的ISBN、存放位置(流通室/阅览室)、状态(不外借/未借出/已借出)、经办人。
    \item \textbf{解释}: 此表记录了图书馆中每本图书的具体信息。
    \item \textbf{3NF分析}: 满足3NF,因为每个非主属性完全函数依赖于主码(图书ID),不存在传递依赖或部分依赖。
    \item \textbf{BCNF分析}: 满足BCNF,因为不存在非主属性对主码的部分或传递依赖,且每个候选码也是超码。
    \item \textbf{4NF分析}: 满足4NF,因为表中不存在多值依赖。
\end{itemize}

\subsubsection{BookReview (书评表)}
\begin{itemize}
    \item \textbf{结构}: 包含读者ID、ISBN、评分、评论、评论时间,其中读者ID和ISBN共同构成复合主码。
    \item \textbf{解释}: 此表用于存储读者对图书的评分和评论。
    \item \textbf{3NF分析}: 满足3NF,因为每个非主属性完全函数依赖于复合主码(读者ID和ISBN),不存在传递依赖或部分依赖。
    \item \textbf{BCNF分析}: 满足BCNF,因为不存在非主属性对复合主码的部分或传递依赖,且每个候选码也是超码。
    \item \textbf{4NF分析}: 满足4NF,因为表中不存在多值依赖。
\end{itemize}

\subsubsection{jsTable (借书表)}
\begin{itemize}
    \item \textbf{结构}: 包含读者ID、图书ID、借阅时间、应还时间、归还时间、记录是否有效,其中读者ID、图书ID和借阅时间共同构成复合主码。
    \item \textbf{解释}: 此表记录了读者借阅图书的详细信息,其中记录是否有效表示借过的书是否仍在库,这样可以避免将借过的书出库后借书记录丢失。
    \item \textbf{3NF分析}: 满足3NF,因为每个非主属性完全函数依赖于复合主码(读者ID、图书ID和借阅时间),不存在传递依赖或部分依赖。
    \item \textbf{BCNF分析}: 满足BCNF,因为不存在非主属性对复合主码的部分或传递依赖,且每个候选码也是超码。
    \item \textbf{4NF分析}: 满足4NF,因为表中不存在多值依赖。
\end{itemize}

\subsection{总结}
所有表均满足第三范式(3NF)和BCNF的要求,
因为它们的每个非主属性都直接依赖于主码(或复合主码),
并且不存在任何传递依赖或部分依赖。此外,所有表也满足第四范式(4NF),
因为它们的表结构比较简单,不存在多值依赖。这样的设计确保了数据的一致性、减少了数据冗余,并且提高了查询效率。
	\section{数据库物理结构设计}
	数据库的物理设计一般依赖于相应的数据库管理系统,由系统自主完成,不用程序员过多考虑,但有时为了效率的问题或其他的要求,必须进行相应的物理设计(包括建立索引),以达到相应的要求。
	由于本图书馆管理系统的数据库表结构较为简单,且目前数据不多,因此暂未建立索引或进行其他物理结构设计,仅使用默认操作。


                            
	\section{后端功能设计与实现(views.py)}
	\subsection{home 函数:主页}
	\begin{enumerate}
        \item \textbf{功能:} home 函数的主要功能是渲染并返回主页模板,处理对网站主页的请求;
        \item \textbf{实现逻辑}:
        \begin{enumerate}
            \item 接收一个请求对象作为参数;
            \item 使用 Django 的 \texttt{render} 函数,将请求对象和指定的模板('home.html')作为参数传递;
            \item 返回渲染后的 HTML 页面给用户。
        \end{enumerate}
    \end{enumerate}
    
    \subsection{login\_view 函数:登录}
    \begin{enumerate}
        \item \textbf{功能:} login\_view 函数处理用户登录逻辑,支持两种用户类型:读者和管理员。根据用户类型和提供的凭证(用户名和密码),进行相应的身份验证,并根据验证结果重定向到不同的页面或返回错误信息;
        \item \textbf{实现逻辑}:
        \begin{enumerate}
            \item 初始化一个空字典 \texttt{context} 用于存储模板变量;
            \item 检查请求方法是否为 POST。如果不是,直接渲染并返回主页模板;
            \item 从 POST 请求中获取用户名、密码和用户类型;
            \item 验证用户名和密码是否提供。如果任一未提供,设置错误消息并重新渲染主页模板,显示错误消息;
            \item 根据用户类型(管理员或读者)执行不同的逻辑:
            \begin{enumerate}
                \item 管理员:使用用户名查询 \texttt{tsglyTable} 表,检查密码是否匹配。如果登录成功,设置会话变量并重定向到管理员首页;
                \item 读者:使用用户名查询 \texttt{dzTable} 表,检查密码是否匹配。如果登录成功,设置会话变量并重定向到读者首页;
            \end{enumerate}
            \item 如果用户名或密码不匹配,设置错误消息并重新渲染主页模板,显示错误消息。
        \end{enumerate}
    \end{enumerate}
    在这个过程中,\texttt{context} 字典被用来传递模板变量(如错误消息),而 Django 的 session 机制被用来在登录成功后存储用户信息,以便于跨请求保持用户状态。
    \subsection{register 函数:读者注册}
\begin{enumerate}
    \item \textbf{功能:} 此函数用于新用户注册账户;
    \item \textbf{实现逻辑}:
    \begin{enumerate}
        \item 初始化一个空字典 \texttt{context} 用于存储传递给模板的数据;
        \item 如果请求方法为 GET,返回注册页面模板;
        \item 如果请求方法为 POST,执行注册逻辑:
        \begin{enumerate}
            \item 从 POST 数据中获取用户输入的姓名(xm)、密码(mm)和密码确认(mmqr);
            \item 验证姓名、密码和密码确认是否都已填写,如果有任何一个未填写,则在 \texttt{context} 中设置错误消息并重新渲染注册页面;
            \item 验证两次输入的密码是否一致,如果不一致,则设置错误消息并重新渲染注册页面;
            \item 验证密码长度是否至少为六位,如果不是,则设置错误消息并重新渲染注册页面;
            \item 检查用户名是否已被使用,如果是,则设置错误消息并重新渲染注册页面;
            \item 如果用户名未被使用,计算新用户的 ID,创建新用户记录,并保存到数据库;
            \item 注册成功后,重定向到登录页面。
        \end{enumerate}
        \item 如果请求方法既不是 GET 也不是 POST,则返回注册页面模板。
    \end{enumerate}
\end{enumerate}

\subsection{logout\_view 函数}
\begin{enumerate}
    \item \textbf{功能:} 此函数用于处理读者和管理员的退出登录操作;
    \item \textbf{实现逻辑}:
    \begin{enumerate}
        \item 检查 session 中是否存在 \texttt{login\_type},如果存在,则清空 session;
        \item 重定向到网站主页。
    \end{enumerate}
\end{enumerate}
\subsection{dz\_index 函数:读者首页}
\begin{enumerate}
    \item \textbf{功能}:显示读者的首页信息,包括当前借阅、历史借阅记录、用户排名等;
    \item \textbf{实现逻辑}:
    \begin{enumerate}
        \item 验证用户登录类型是否为读者,如果不是,则重定向到主页;
        \item 初始化上下文字典 \texttt{context},并从会话中获取用户姓名和ID,存入上下文;
        \item 查询当前用户的所有借阅记录,并筛选出当前借阅记录(未归还的);
        \item 计算总用户数;
        \item 基于借阅数量计算当前用户的排名;
        \item 遍历所有借阅记录,提取书籍信息和借阅状态,存入列表 \texttt{grzt};
        \item 使用 \texttt{Paginator} 对 \texttt{grzt} 进行分页处理;
        \item 从请求中获取页码,获取对应的页面对象 \texttt{page\_obj},并将其及其他信息添加到上下文;
        \item 渲染并返回 \texttt{dz\_index.html} 页面。
    \end{enumerate}
\end{enumerate}

\subsection{current\_borrows\_view 函数:当前借阅书籍}
\begin{enumerate}
    \item \textbf{功能}:显示读者当前借阅的书籍;
    \item \textbf{实现逻辑}:
    \begin{enumerate}
        \item 验证用户登录类型是否为读者,如果不是,则重定向到主页;
        \item 初始化上下文字典 \texttt{context},并从会话中获取用户姓名和ID,存入上下文;
        \item 查询当前用户的所有未归还借阅记录;
        \item 遍历当前借阅记录,提取书籍信息和借阅状态,存入列表 \texttt{grzt};
        \item 使用 \texttt{Paginator} 对 \texttt{grzt} 进行分页处理;
        \item 从请求中获取页码,获取对应的页面对象 \texttt{page\_obj},并将其及其他信息添加到上下文;
        \item 渲染并返回 \texttt{current\_borrows.html} 页面。
    \end{enumerate}
\end{enumerate}

\subsection{book\_details 函数:书籍详情}
\begin{enumerate}
    \item \textbf{功能}:显示一本书的详细信息,包括书籍信息和书评;
    \item \textbf{实现逻辑}:
    \begin{enumerate}
        \item 使用 \texttt{isbn} 参数从 \texttt{tsTable} 表中过滤出书籍信息,存储在 \texttt{books\_info} 变量中;
        \item 使用 \texttt{isbn} 参数从 \texttt{BookReview} 表中过滤出评分,并计算平均值,如果没有评分则默认为 0,结果存储在 \texttt{average\_score} 变量中;
        \item 使用 \texttt{isbn} 参数从 \texttt{BookReview} 表中过滤出所有评论,存储在 \texttt{reviews} 变量中;
        \item 创建一个字典 \texttt{context},包含用户会话中的姓名和 ID,书籍信息,平均评分,和评论列表;
        \item 使用 \texttt{render} 函数渲染 \texttt{book\_details.html} 模板,并传递 \texttt{context} 字典.
    \end{enumerate}
\end{enumerate}
\subsection{dz\_smztcx 函数:读者书目状态查询}
\begin{enumerate}
    \item \textbf{功能}:允许读者查询书目状态,包括书籍的详细信息和书评提交功能;
    \item \textbf{实现逻辑}:
    \begin{enumerate}
        \item 验证用户登录类型是否为读者,如果不是,则重定向到主页;
        \item 初始化上下文字典 \texttt{context},并从会话中获取用户姓名和ID,存入上下文;
        \item 如果是POST请求且包含书评信息,则处理书评提交:
        \begin{enumerate}
            \item 从POST请求中获取ISBN码、评分和评论内容;
            \item 尝试根据ISBN码在 \texttt{smTable} 表中找到对应的书籍实例;
            \item 如果找到,则在 \texttt{BookReview} 表中创建一条新的书评记录,并返回成功消息;
            \item 如果未找到对应的书籍实例,则返回错误消息;
        \end{enumerate}
        \item 如果是GET请求,则直接渲染并返回 \texttt{dz\_smztcx.html} 页面;
        \item 如果是POST请求但不是提交书评的请求,则处理书目查询:
        \begin{enumerate}
            \item 从POST请求中获取书名、作者、ISBN码和出版社信息,并存入上下文;
            \item 如果书名为空,则返回错误消息并提示输入书名进行搜索;
            \item 使用提供的信息对 \texttt{smTable} 表进行过滤查询,实现模糊搜索;
            \item 遍历查询结果,提取每本书的详细信息和状态,存入列表 \texttt{smzt};
            \item 将 \texttt{smzt} 列表和其他信息添加到上下文;
            \item 渲染并返回 \texttt{dz\_smztcx.html} 页面。
        \end{enumerate}
    \end{enumerate}
\end{enumerate}

\subsection{check\_review 函数:检查是否已经评论过}
\begin{enumerate}
    \item \textbf{功能}:检查用户是否已经对某本书进行了评论;
    \item \textbf{实现逻辑}:
    \begin{enumerate}
        \item 从会话中获取用户ID;
        \item 使用用户ID和ISBN码查询 \texttt{BookReview} 表,检查是否存在评论记录;
        \item 如果存在评论记录,返回提示信息,说明用户已经评论过该书;
        \item 如果不存在评论记录,返回空消息,允许用户进行评论;
        \item 返回包含检查结果和消息的JSON响应。
    \end{enumerate}
\end{enumerate}

\subsection{submit\_review 函数:提交评论}
\begin{enumerate}
    \item \textbf{功能}:允许用户提交对书籍的评论;
    \item \textbf{实现逻辑}:
    \begin{enumerate}
        \item 检查请求方法是否为POST,如果不是,返回错误消息;
        \item 从POST请求中获取ISBN码、评分和评论内容;
        \item 从会话中获取用户ID;
        \item 如果评分为空,返回错误消息,说明评分是必填项;
        \item 尝试根据ISBN码在 \texttt{smTable} 表中找到对应的书籍实例;
        \item 如果找到书籍实例,创建一条新的书评记录,并返回成功消息;
        \item 如果未找到书籍实例,返回错误消息,说明提供的ISBN码无效;
    \end{enumerate}
\end{enumerate}

\subsection{dz\_js 函数:读者借书}
\begin{enumerate}
    \item \textbf{功能}:实现读者借书的功能;
    \item \textbf{实现逻辑}:
    \begin{enumerate}
        \item 验证用户登录类型是否为读者,如果不是,则重定向到主页;
        \item 初始化上下文字典 \texttt{context},并从会话中获取用户姓名和ID,存入上下文;
        \item 如果是GET请求,直接渲染并返回 \texttt{dz\_js.html} 页面;
        \item 如果是POST请求,处理借书逻辑:
        \begin{enumerate}
            \item 从POST请求中获取ISBN码,并存入上下文;
            \item 如果ISBN码为空,返回错误消息并提示填写完整的ISBN号;
            \item 尝试根据ISBN码在 \texttt{smTable} 表中找到对应的书籍实例;
            \item 如果未找到书籍实例,返回错误消息,说明ISBN号填写错误;
            \item 检查当前用户的未归还借阅记录数是否已达上限;
            \item 如果已达上限,返回错误消息,说明借阅书籍数已达上限;
            \item 检查所选图书的状态是否为“未借出”;
            \item 如果所有图书已被借出,返回错误消息,说明无法借阅;
            \item 更新图书状态为“已借出”,并创建借书记录;
            \item 返回成功消息,提示借阅成功,并显示图书ID,借阅时间为60天,不支持续借。
        \end{enumerate}
    \end{enumerate}
\end{enumerate}
\subsection{dz\_hs 函数:读者还书}
\begin{enumerate}
    \item \textbf{功能}:实现读者还书的功能;
    \item \textbf{实现逻辑}:
    \begin{enumerate}
        \item 验证用户登录类型是否为读者,如果不是,则重定向到主页;
        \item 初始化上下文字典 \texttt{context},并从会话中获取用户姓名和ID,存入上下文;
        \item 如果是GET请求,直接渲染并返回 \texttt{dz\_hs.html} 页面;
        \item 如果是POST请求,处理还书逻辑:
        \begin{enumerate}
            \item 从POST请求中获取图书ID,并存入上下文;
            \item 如果图书ID为空或不是数字,返回错误消息;
            \item 查询图书ID是否存在,如果不存在,返回错误消息;
            \item 查询该读者是否有未归还的借书记录,如果没有,返回错误消息;
            \item 检查图书是否逾期未还,如果逾期,计算并提示逾期费用;
            \item 更新图书状态为“未借出”,并记录归还时间;
            \item 返回成功消息,提示图书已归还。
        \end{enumerate}
    \end{enumerate}
\end{enumerate}

\subsection{my\_reviews 函数:读者评书记录}
\begin{enumerate}
    \item \textbf{功能}:显示读者的评书记录;
    \item \textbf{实现逻辑}:
    \begin{enumerate}
        \item 从会话中获取读者ID,并初始化上下文字典 \texttt{context};
        \item 查询当前读者的所有评书记录,并按评论时间排序;
        \item 使用 \texttt{Paginator} 对评书记录进行分页处理;
        \item 从请求中获取页码,获取对应的页面对象 \texttt{page\_obj},并将其及其他信息添加到上下文;
        \item 遍历分页后的评书记录,提取书籍信息、评分、评论内容等,存入列表;
        \item 渲染并返回 \texttt{my\_reviews.html} 页面。
    \end{enumerate}
\end{enumerate}

\subsection{revoke\_review 函数:撤销评论}
\begin{enumerate}
    \item \textbf{功能}:允许读者撤销自己的评论;
    \item \textbf{实现逻辑}:
    \begin{enumerate}
        \item 从POST请求的body中解析JSON数据,获取评论ID;
        \item 尝试根据评论ID找到对应的评论记录;
        \item 如果找到,删除该评论记录,并返回成功消息;
        \item 如果未找到或发生其他错误,返回错误消息。
    \end{enumerate}
\end{enumerate}
\subsection{ranking 函数:展示排名信息}
\begin{enumerate}
    \item \textbf{功能}:展示评分最高的十本书及其评分,以及被借阅次数最多的十本书及其被借次数。
    \item \textbf{实现逻辑}:
    \begin{enumerate}
        \item 验证用户登录类型是否为读者,如果不是,则重定向到主页;
        \item 初始化上下文字典 \texttt{context},并从会话中获取用户姓名和ID,存入上下文;
        \item 获取评分最高的十本书及其评分,确保书籍有评分,存入上下文;
        \item 获取被借阅次数最多的十本书及其被借次数,确保书籍被借阅过,存入上下文;
        \item 渲染并返回 \texttt{ranking.html} 页面,展示排名信息。
    \end{enumerate}
\end{enumerate}

\subsection{gly\_index 函数:管理员首页}
\begin{enumerate}
    \item \textbf{功能}:显示管理员首页。
    \item \textbf{实现逻辑}:
    \begin{enumerate}
        \item 验证用户登录类型是否为管理员,如果不是,则重定向到主页;
        \item 初始化上下文字典 \texttt{context},并从会话中获取管理员姓名和ID,存入上下文;
        \item 渲染并返回 \texttt{gly\_index.html} 页面,显示管理员首页。
    \end{enumerate}
\end{enumerate}

\subsection{book\_details2 函数:管理员页面书籍详情}
\begin{enumerate}
    \item \textbf{功能}:在管理员页面显示书籍的详细信息。
    \item \textbf{实现逻辑}:
    \begin{enumerate}
        \item 根据传入的ISBN查询书籍信息;
        \item 查询并计算书籍的平均评分;
        \item 查询书籍的所有评论;
        \item 初始化上下文字典 \texttt{context},并将查询到的信息存入上下文;
        \item 渲染并返回 \texttt{book\_details2.html} 页面,展示书籍详情。
    \end{enumerate}
\end{enumerate}

\subsection{gly\_smztcx 函数:管理员书目状态查询}
\begin{enumerate}
    \item \textbf{功能}:允许管理员根据书名、作者、ISBN、出版社等信息查询书目状态。
    \item \textbf{实现逻辑}:
    \begin{enumerate}
        \item 验证用户登录类型是否为管理员,如果不是,则重定向到主页;
        \item 初始化上下文字典 \texttt{context},并从会话中获取管理员姓名和ID,存入上下文;
        \item 如果是GET请求,直接渲染并返回 \texttt{gly\_smztcx.html} 页面;
        \item 如果是POST请求,处理书目状态查询逻辑:
        \begin{enumerate}
            \item 从POST请求中获取书名、作者、ISBN、出版社等信息;
            \item 如果书名为空,返回错误消息;
            \item 使用模糊搜索查询满足条件的书目,并计算每本书的在库册数、不外借册数、未借出册数、已借出册数;
            \item 将查询结果和其他信息存入上下文;
            \item 渲染并返回 \texttt{gly\_smztcx.html} 页面,展示查询结果。
        \end{enumerate}
    \end{enumerate}
\end{enumerate}
\subsection{smzt\_all 函数:所有书目状态查询}
\begin{enumerate}
    \item \textbf{功能}:显示所有书目的状态信息;
    \item \textbf{实现逻辑}:
    \begin{enumerate}
        \item 初始化上下文字典 \texttt{context},并从会话中获取管理员ID,存入上下文;
        \item 查询所有书目的信息;
        \item 遍历查询结果,对每本书进行以下操作:
        \begin{enumerate}
            \item 提取书目的基本信息(ISBN、书名、作者、出版社、出版年月);
            \item 查询并计算该书目的库存总数、不外借数量、未借出数量、已借出数量;
            \item 将上述信息存入列表 \texttt{smzt} 中;
        \end{enumerate}
        \item 使用 \texttt{Paginator} 对 \texttt{smzt} 进行分页处理,每页显示10个书目;
        \item 从请求中获取页码,获取对应的页面对象 \texttt{page\_obj},并将其及其他信息添加到上下文;
        \item 渲染并返回 \texttt{smzt\_all.html} 页面。
    \end{enumerate}
\end{enumerate}

\subsection{borrowed\_books 函数:所有借阅信息}
\begin{enumerate}
    \item \textbf{功能}:显示所有当前借阅信息;
    \item \textbf{实现逻辑}:
    \begin{enumerate}
        \item 验证用户登录类型是否为管理员,如果不是,则重定向到主页;
        \item 初始化上下文字典 \texttt{context},并从会话中获取用户姓名和ID,存入上下文;
        \item 查询所有当前未归还的借阅记录,并按读者ID排序;
        \item 将查询结果存入上下文;
        \item 渲染并返回 \texttt{borrowed\_books.html} 页面。
    \end{enumerate}
\end{enumerate}

\subsection{gly\_rk 函数:管理员入库}
\begin{enumerate}
    \item \textbf{功能}:管理员进行书籍入库操作;
    \item \textbf{实现逻辑}:
    \begin{enumerate}
        \item 验证用户登录类型是否为管理员,如果不是,则重定向到主页;
        \item 初始化上下文字典 \texttt{context},并从会话中获取用户姓名和ID,存入上下文;
        \item 如果请求方法为GET,直接渲染并返回 \texttt{gly\_rk.html} 页面;
        \item 如果请求方法为POST,执行以下操作:
        \begin{enumerate}
            \item 从请求中获取ISBN、入库数量、入库后状态、书名、作者、出版社、出版年月等信息,并存入上下文;
            \item 验证必要信息的完整性,不完整则返回错误信息;
            \item 根据ISBN查询书目,判断是旧书入库还是新书入库;
            \item 对于旧书入库,进一步验证书名、作者、出版社、出版年月的匹配性,不匹配则返回错误信息;
            \item 根据入库后状态(流通室、阅览室),创建相应数量的图书实例,并保存;
            \item 更新书目的库存总数,并保存;
            \item 设置成功信息,并可能刷新页面;
            \item 渲染并返回 \texttt{gly\_rk.html} 页面,显示操作结果。
        \end{enumerate}
    \end{enumerate}
\end{enumerate}
\subsection{gly\_ck 函数:管理员出库}
\begin{enumerate}
    \item \textbf{功能}:管理员进行书籍出库操作;
    \item \textbf{实现逻辑}:
    \begin{enumerate}
        \item 验证用户登录类型是否为管理员,如果不是,则重定向到主页;
        \item 初始化上下文字典 \texttt{context},并从会话中获取用户姓名和ID,存入上下文;
        \item 如果请求方法为GET,直接渲染并返回 \texttt{gly\_ck.html} 页面;
        \item 如果请求方法为POST,执行以下操作:
        \begin{enumerate}
            \item 从请求中获取ISBN、出库数量、出库优先位置,并存入上下文;
            \item 验证必要信息的完整性,不完整则返回错误信息;
            \item 验证出库优先位置是否合法(流通室或阅览室),不合法则返回错误信息;
            \item 根据ISBN查询书目,如果不存在则返回错误信息;
            \item 计算未借出和不外借的图书数量,以及所有图书数量;
            \item 如果出库数量超过藏书总数或可出库数量,返回错误信息;
            \item 根据出库优先位置,优先出库未借出或不外借的图书;
            \item 更新图书状态为无效,并删除图书记录;
            \item 更新书目的库存总数,如果没有剩余的书,则删除书目记录;
            \item 设置成功信息,并返回出库成功的图书ID;
            \item 渲染并返回 \texttt{gly\_ck.html} 页面,显示操作结果。
        \end{enumerate}
    \end{enumerate}
\end{enumerate}

\subsection{book\_count\_view 函数:书籍借阅次数统计}
\begin{enumerate}
    \item \textbf{功能}:统计并显示各书籍的借阅次数;
    \item \textbf{实现逻辑}:
    \begin{enumerate}
        \item 从 \texttt{jsTable} 中查询每本书的ISBN、书名和借阅次数;
        \item 对查询结果按借阅次数降序排序;
        \item 将查询结果存入上下文字典 \texttt{context};
        \item 渲染并返回 \texttt{book\_count.html} 页面。
    \end{enumerate}
\end{enumerate}

\subsection{reader\_count\_view 函数:读者借阅次数统计}
\begin{enumerate}
    \item \textbf{功能}:统计并显示各读者的借阅次数;
    \item \textbf{实现逻辑}:
    \begin{enumerate}
        \item 从 \texttt{jsTable} 中查询每位读者的ID、姓名和借阅次数;
        \item 对查询结果按借阅次数降序排序;
        \item 将查询结果存入上下文字典 \texttt{context};
        \item 渲染并返回 \texttt{reader\_count.html} 页面。
    \end{enumerate}
\end{enumerate}
	\section{一些bug及修改思路}
	\subsection{管理员注册}
    一开始管理员和读者注册的逻辑是一样的,但是经助教提醒,发现这样会导致信息的泄露。
    助教建议我们在新管理员注册后需要得到老管理员的批准,但我们发现这样实现起来比较困难,
    因此我们决定使用超级管理员总结在数据库中添加图书管理员的方式来实现管理员注册。
    \subsection{views和templates的对应关系}
    有几次views传入的参数不足,导致渲染模板时出现错误,但报错信息并不明显,导致花了很多时间来找错误。
    \section{待完善的功能}
    \begin{itemize}
        \item \textbf{图书图片展示}:由于时间原因,我们没有在书籍详情页面实现图书图片的展示;
        \item \textbf{图书分类和搜索}:在实际应用中,用户可能需要根据图书的分类或关键字进行搜索,这是一个可以增加系统实用性的功能;
        \item \textbf{图书借阅提醒}:用户借阅图书后,系统可以通过邮件或短信提醒用户还书日期,这是一个可以提高系统友好性的功能,但涉及到其他较为复杂的操作,暂时没有完成;
        \item \textbf{图书预约}:用户可以预约已被借出的图书,系统会在图书归还后通知该用户,且其他用户不能借阅被预约的书,这是一个可以提高系统服务质量的功能;
        \item \textbf{图书续借}:用户可以对已借阅的图书进行续借操作,延长借阅时间;
        \item \textbf{图书荐购}:用户可以向管理员推荐购买某本书,管理员可以根据用户的推荐进行购买决策;
        \item \textbf{图书推荐}:根据用户的借阅历史和评分记录,系统可以推荐用户可能感兴趣的图书,这是一个可以提高系统个性化的功能;
        \item \textbf{图书分类统计}:管理员可能需要查看每个分类的图书数量,以便更好地管理图书馆的藏书;
        \item \textbf{图书借阅统计}:管理员可能需要查看每本书的借阅次数,以便更好地了解读者的阅读喜好;
        \item \textbf{入库图书统计}:管理员可能需要查询自己入库的所有书籍。
    \end{itemize} 
    \section{总结}
	\subsection{实验收获}

	在本次实验中,我们成功设计并实现了一个图书馆管理系统的基本功能。通过此次实验,我们收获了以下几点:
	
	\begin{itemize}
		\item \textbf{数据库设计与范式理论应用}:
		\begin{itemize}
			\item 通过具体实践,我们学会了如何将业务需求转化为数据库表结构,并且掌握了表之间关系的设计,包括一对一、一对多、多对多等关系;
			\item 我们也通过对每个表的3NF、BCNF和4NF的分析,更加深刻地理解了数据库设计的范式理论,以及如何通过范式理论减少数据冗余和提高数据一致性。
		\end{itemize}

		
		\item \textbf{Django ORM(Object-Relational Mapping)工具的使用}:
		\begin{itemize}
			\item 学习了如何使用Django的ORM来定义数据库模型,并且通过这些模型与数据库进行交互,简化了数据操作的复杂度。
			\item 掌握了如何在Django中进行数据迁移,维护数据库表结构的变更。
		\end{itemize}
		
		\item \textbf{团队合作}:
		\begin{itemize}
			\item 通过分工合作,培养了团队合作能力,提高了沟通和协调能力,确保了项目的顺利进行。
		\end{itemize}
        \item \textbf{持续学习}:
        \begin{itemize}
            \item 数据库技术在不断进步,新的设计理念和技术工具不断涌现。我们需要保持学习的态度,不断更新知识,以适应技术发展的需要。
		\end{itemize}
	\end{itemize}
	
	\subsection{一些问题和教训}
	
	\begin{itemize}
		\item \textbf{需求分析不够全面}:
		\begin{itemize}
			\item 在初期的需求分析阶段,由于考虑不够全面,导致后期需要对数据库表结构进行多次调整。这提醒我们在进行系统设计前,需要进行更为详细和全面的需求分析。
		\end{itemize}
		
		\item \textbf{表结构调整的复杂性}:
		\begin{itemize}
			\item 由于初始设计的一些缺陷,后期在对表结构进行调整时,涉及到的数据迁移和数据一致性问题较为复杂。这也提醒我们在设计数据库时,需要充分考虑未来可能的变化和扩展。
		\end{itemize}
		
		\item \textbf{数据量和性能问题}:
		\begin{itemize}
			\item 数据库设计中的每一个细节都可能影响到系统的性能和稳定性,虽然当前系统数据量较少,但随着数据量的增加,某些查询的效率可能会下降。这提示我们在实际应用中,需要考虑性能优化的问题,如建立索引、优化查询等。
		\end{itemize}
		\item \textbf{测试不足}:
		\begin{itemize}
			\item 在功能实现后,对系统的测试可能不够全面,不能保证所有功能在所有情况下都不会出现问题。
		\end{itemize}
        \item \textbf{没写实验日志}:
        \begin{itemize}
			\item 由于是初次完成这种较大的项目,因此没有记录完成每部分内容的时间点以及遇到的问题和解决方案,导致后期写实验报告时有些内容忘记了。
		\end{itemize}
	\end{itemize}
    
\end{document}