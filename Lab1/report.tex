\documentclass{ctexart}
\usepackage{graphicx}
\usepackage{caption}
\usepackage{float}
\usepackage{amsmath}
\usepackage{fancyhdr}
\usepackage{xunicode-addon}
\usepackage{booktabs}
\usepackage[a4paper,hmargin=1.25in,vmargin=1in]{geometry}
% !TeX program = xelatex
\title{\begin{figure}[H]
	\centering 
	\includegraphics[height=7cm,width=14cm]{E:/Pictures/中科大.jpg}
	\end{figure}\Huge\textbf{Lab 1}\\\huge{SQL 练习}}
\date{}
\punctstyle{banjiao} 
\pagestyle{fancy}
	\fancyhead[C]{\LARGE\textbf{Lab 1}}
	\fancyhead[L]{}
	\fancyhead[R]{}
	\fancyfoot[C]{\thepage}
\begin{document}
	\maketitle
	\thispagestyle{empty}
	
	\[\makebox{\Large{姓名:\underline{\makebox[5cm]{高茂航}}}}\]
	
    \[\makebox{\Large{学号:\underline{\makebox[5cm]{PB22061161}}}}\]
	
	$$\makebox{\Large{日期:\underline{\makebox[5cm]{2024.4.20}}}}$$
	
	\clearpage

	\pagenumbering{arabic}
%\section{Results}
设有如下关系模式:
Student(SNO, NAME, GENDER, BIRTHDAY, DEPART)
SNO 为学生学号, DEPART 为系别
Teacher(TNO,NAME,GENDER,BIRTHDAY,POSITION,DEPART)
TNO 为教师工号, POSITION 为职称, DEPART 为系别。
Course(CNO, NAME, TYPE, TNO)
CNO 为课程编号, TYPE 为课程类型, 1 表示必修课, 0 表示选修课, TNO 为教师工号。
Score(SNO, CNO, DEGREE)
SNO 为学生学号, CNO 为课程编号, DEGREE 为成绩。
PS:加下划线"\_\_"的表示该字段为主键。
\section{任务一}
根据所给 Student.csv、 Teacher.csv、 Course.csv、 Score.csv 表中的数据信息,在数
据库中创建对应的关系表并将数据录入到数据库中。可能涉及到的数据类型: varchar, char,
int, float ,datetime。
(也需要截图展示! )

\section{任务二:实现以下各题功能的 SQL 语句(部分题目的结果受前面题目影响)}

\subsection{修改基本表}
\subsubsection{在学生表 student 中增加一个新的属性列 AGE(年龄),类型为 int}
\subsubsection{计算每个学生的年龄(AGE,即当前年份减去出生年份)}
注意,此操作可能需要关闭安全更新模式;提示,可使用 MySQL 的 YEAR 函数。
\subsubsection{为每个学生的年龄加 2, 将 AGE(年龄)的数据类型由 int 改为 char}
\subsubsection{删除属性列 AGE}
\subsubsection{创建一个教师课程数量表: teacher\_course(TNO,NUM\_COURSE),两个属性分别表示授课教师工号,课程数量, 其中 TNO 是主键}
\subsubsection{用一条语句,结合表 course 记录,为表 teacher 中所有教师,在表 teacher\_course 添加对应记录(用到查询,而不是手动数 NUM 插入; 若是表 course 中未出现的教师,则课程数量记为 NULL)}
\subsubsection{删除表 teacher\_course 中含有 NULL 的记录}
\subsubsection{删除表 teacher\_course}
\subsubsection{在学生表 student 、成绩表 score 中分别插入一些数据,数据如下(注意: 如果与原有数
据冲突,比如学号重复,请修改一下自己的信息保证能够插入):}
请保证选择的三门课成绩不一样。
\subsubsection{在 score 表中删除你所选的课程中成绩最低的一门课程的记录(可能会用到子查询)}
\subsection{索引}
\subsubsection{用 create 语句在 course 表的名称 NAME 上建立普通索引 NAME\_INDEX}
\subsubsection{用 create 语句在 teacher 表的工号 TNO 上建立唯一索引 TNO\_INDEX}
\subsubsection{用 create 语 句 在 score 表 上 的 学 号 SNO 、 成 绩 DEGREE 上 建 立 复 合 索引RECORD\_INDEX, 要求学号为降序,学号相同时成绩为升序}
\subsubsection{用一条语句查询表 score 的索引}
\subsubsection{删除 teacher 表字段 TNO 上的索引 TNO\_INDEX}
\subsection{查询}
注:以下每道题限制用一条 SQL 语句完成
\subsubsection{查询和你属于同一个系的学生学号和姓名(包括你本人)}
\subsubsection{查询和你属于同一个系的学生学号和姓名(不包括你本人)}
\subsubsection{查询和你的某个好友属于同一个系的学生学号和姓名(9 题插入的某个好友)}
\subsubsection{查询和你的两个好友都不在一个系的学生学号和姓名(9 题插入的两个好友)}
\subsubsection{查询教过你的所有老师的工号和姓名}
\subsubsection{查询 11 系和 229 系教师的总人数}
\subsubsection{查询你的系中年龄(即当前年份减去出生年份) 最大的学生的学号、姓名和年龄}
\subsubsection{查询你的系中年龄(即当前年份减去出生年份) 最小的学生的学号、姓名和年龄}
\subsubsection{查询选修 DB\_Design 课程且成绩在 75 分以下(不包括 75) 的学生的学号、姓名和分数}
\subsubsection{查询选修过“ZDH”老师课程的学生学号和姓名(去掉重复行)}
\subsubsection{查询选过某课程的学生学号和分数,并按分数降序展示}
(某课程是指 course 表中的某一课程名 NAME,你自行选择; 分数指的是选的这门课的成绩,不包括选这门课的同学的其他
成绩)。
\subsubsection{查询每门课的平均成绩,其中每行包含课程号、课程名和平均成绩(包括平均成绩为NULL,即该课没有成绩)}
\subsubsection{查询每门必修课的平均成绩,其中每行包含课程号、课程名和平均成绩(包括平均成绩为NULL,即该课没有成绩)}
\subsubsection{查询至少选修了 ZDH 老师(TNO=”TA90023”)开设的所有课程的学生学号}
\subsubsection{查询每门课程的最高分和最低分,并计算其分数差。其中每行包含课程号、课程名和最高分、最低分和分数差(课程无成绩的不用包括)}
\subsubsection{查询存在考试成绩低于 75 分的学生学号,以及低于 75 分的课程数量}
\subsubsection{查询所教过的课程中有学生考试成绩低于 75 分的教师的工号和姓名(去掉重复行)}
\subsubsection{查询选修少于 2 门课程的学生的学号、姓名}
\subsubsection{查询至少选修了 YH 同学(SNO=”PB210000001”)所选全部课程的学生学号}
\subsubsection{查询 Course 表中各个课程名称与相应的平均成绩(没有选课的学生平均成绩为 NULL) }
\subsubsection{查询每个系的学生人数和每个系的平均分,其中每行包含系号、系的人数和平均成绩}
(计算人数的时候需要包括那些没有成绩的同学,但是计算成绩的时候不需要包括这些同学)
\subsubsection{查询所有未选修 DB\_Design 课程或者 Data\_Mining 课程的学生的学生姓名(去掉重复行)}
\subsubsection{查询各个课程的课程名及选该课的学生的最小年龄、最大年龄和平均年龄。(包括没有人选
的课)}
\subsubsection{查询选修了课程名中包含”Computer”课程的学生的学号和姓名}
\subsubsection{设课程平均成绩为 x,查询各个课程成绩处于[x-12, x+12]区间的同学的成绩表,即包含 SNO、CNO、 DEGREE}
\subsection{视图}
\subsubsection{建立 229 系的学生视图(db\_22\_student),属性与 student 表一样,并要求对该视图进行修改和插入操作时仍需保证该视图只有 229 系的学生}
\subsubsection{将 229 系学生视图(db\_229\_student)中学号为“PB210000020”的学生姓名改为{你的姓名(英文首字母)}}
\subsubsection{在 229 系学生视图(db\_229\_student)中找出年龄小于 22 岁的学生,包含 SNO、 NAME}
\subsubsection{向 student 表中插入一名“学号 SA210110021,姓名 QXY,性别女,生日 2007/7/27, 229系”的学生。然后查询视图 db\_229\_student 的所有学生,验证其是否更新}
\subsubsection{向视图 db\_229\_student 中插入一名“学号 SA210110023,姓名 DPC,性别男,生日1997/4/27, 11 系”的学生,观察到了什么现象}
\subsubsection{删除视图 db\_229\_student}
\subsection{触发器}
\subsubsection{创建关系表: teacher\_salary(TNO, SAL),其中 TNO 是教师工号(主键), SAL 是教师工资(类型 float)}
\subsubsection{定义一个 BEFORE 行级触发器,为关系表 teacher\_salary 定义完整性规则: “表中出现的工号必须也出现在 teacher 表中,否则报错”。}
注:该规则实际上就是外键约束; MySQL 中
可使用 SIGNAL 抛出错误;需要为 INSERT 和 UPDATE 分别定义触发器。请展示出成功创建
触发器和测试抛出错误信息的截图。
\subsubsection{定义一个 BEFORE 行级触发器,为关系表 teacher\_salary 定 义 完 整 性 规 则 :
“Instructor/Associate Professor/Professor 的工资不能低于 4000/7000/10000,且不能高于
7000/10000/13000,如果低于,则改为 4000/7000/10000”,如果高于,则改为 7000/10000/13000}
注:需要为 INSERT 和 UPDATE 分别定义触发器。并检验触发器是否工作:为 teacher\_salary构造 INSERT 和 UPDATE 语句并激活所定义过的触发器,将过程截图展示
\subsubsection{删除刚刚创建的所有触发器}
\subsection{空值}
\subsubsection{将 score 表中的 Data\_Mining 课程成绩设为空值,然后在 score 表查询学生学号和分数,并按分数升序展示。观察 NULL 在 MySQL 中的大小是怎样的}
\subsection{开放题}
\subsubsection{}
\subsubsection{}

	\section{Conclusion}
    \end{document}